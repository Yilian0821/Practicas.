\documentclass[11pt]{beamer}
\usepackage[utf8]{inputenc}
\usepackage[T1]{fontenc}
\usepackage{lmodern}
\usepackage[spanish]{babel}
\usepackage{graphics}
\usetheme{CambridgeUS}
\usepackage{rrgtrees}




\begin{document}
\author[Aprendiendo latex]{Yilian Vazquez Martinez }
\title{Análisis de Productos }
\date{7/2023}
\subtitle{cerveza, refresco y cebolla}
%\logo{}
\institute{
	\inst{1}
	Universidad de la Habana\\ 
	Facultad: Matcom.
}
%\date{}
%\subject{}
%\setbeamercovered{transparent}
%\setbeamertemplate{navigation symbols}{}


\begin{frame}

    \maketitle

	\end{frame}
	
	\begin{frame}{Contenido}
	    \tableofcontents
	\end{frame}
	\section{ Cerveza}
    	\begin{frame}{ Cerveza}
    		
    	    Primeramente investiguemos un poco acerca de la diferencia que existe entre la cerveza en lata o en botella.
    		Básicamente, lo que ocurre es que la botella, al tener un cuello muy estrecho, permite que se pueda desplazar fácilmente el oxigeno y que se pueda taponear rápido sin que haya presencia de este "enemigo" de la cerveza. Por el contrario, la lata tiene una abertura muy grande y esto dificulta mucho el momento de taponearla en ausencia absoluta de oxígeno. Esto provoca que el proceso de oxidación de la cerveza en lata sea mucho mas rápido, podemos decir hasta el momento que la botella conserva un mejor sabor al paladar que la lata. Ahora bien, lo que lleva a las cerveceras a seguir utilizando un formato que conserva el producto peor que el vidrio, es la comodidad a la hora de transportarla, apilarla y almacenarla. En cuanto a la temperatura y la luz , las botellas dejan pasar demasiada luz y las latas no dejan pasar nada. Además la lata se calienta y se enfría más rápido que la botella esto dependiendo de lo queramos en cada momento, puede ser bueno o malo, y ahora tú, que prefieres,  lata o botella?
    	    https://www.directoalpaladar.com/cata-de-cerveza/cerveza-mejor-lata-botella-vidrio
    		Ahora volviendo al análisis de la grafica de barra se puede observar que predomina la lata, y además el color verde. Y es que si miramos bien la grafica de abajo de: Mayores exportadores de cerveza por continente, casi toda la cerveza importada es de Europa, esto responde, a porque casi toda la cerveza que se vende en Guanabacoa es en lata. Ya que con la información anterior, lo atractivo de la lata es la comodidad para transportarla, apilarla y almacenarla.
        \end{frame}
    
        \begin{frame}{Cerveza}
        		Ahora volviendo al análisis de la grafica de barra se puede observar que predomina la lata, y además el color verde. Y es que si miramos bien la grafica de abajo de: Mayores exportadores de cerveza por continente, casi toda la cerveza importada es de Europa, esto responde, a porque casi toda la cerveza que se vende en Guanabacoa es en lata. Ya que con la información anterior, lo atractivo de la lata es la comodidad para transportarla, apilarla y almacenarla.
        \end{frame}
    \section{Refresco}
	    \begin{frame}{Refresco}
	            La obesidad está relacionada con padecimientos crónicos, como la diabetes, hipertensión, cáncer, enfermedades cardiovasculares, entre otros. Por lo anterior, la dependencia recomienda consumir refresco con moderación para evitar problemas en la salud. Por lo que también considera importante dar a conocer cuáles son las bebidas gaseosas con menos azúcares
	            De acuerdo con los resultados del estudio publicados en mayo de este año, las mejores marcas de refresco, por contener menos azúcares, son las siguientes:
	            
	            . Sidral Mundet y Sprite
	            . Fresca
	            . Fanta
	            . Delaware Punch
	            . Manzanita Sol
	            
	    \end{frame}
	
    	\begin{frame}{Refresco}
 	   ¿Por qué son las mejores? por cada envase de 355 mililitros, dichas marcas contienen estos gramos de azúcares:
 	   
 	   .Sidral Mundet y Sprite: ambos contienen 20.2 gramos.
 	   .Fresca: 19.5 g
 	   .Fanta: 19.2 g
 	   .Delaware Punch: 18.5 g
 	   .Manzanita Sol: 17.8 g
 	   
 	   Antes de comprar un refresco en la tiendita de la esquina o en el supermercado,se recomienda revisar la fecha de caducidad y sus niveles de azúcares, principalmente si padeces una enfermedad crónica.
 	   
        \end{frame}
    \section{Cebolla}
	    \begin{frame}{Cebolla}
		¿Cuál es la diferencia entre la cebolla morada y la blanca? Si bien ambas tienen el mismo sabor y forma, hay algunas diferencias importantes que vale la pena conocer. La cebolla morada es un poco más dulce y menos picante que la cebolla blanca. Esto significa que se puede comer cruda en ensaladas o sándwiches sin causar lágrimas. Además, la cebolla morada se puede utilizar para añadir color a un plato.
		En general, la cebolla blanca se cocina mejor que la cebolla morada. Esto se debe a que la cebolla blanca se vuelve más suave y dulce al cocinarla, mientras que la cebolla morada se vuelve más amarga. También, la cebolla blanca es más adecuada para usar como sabor base en sopas, arroces, guisos y otros platos.
		La cebolla morada es mejor para añadir sabor a platos a los que se desea añadir color. Estas cebollas son excelentes para decorar platos o servir crudas en ensaladas. También pueden ser agregadas a salsas o condimentos para añadir sabor y color.
	    \end{frame}
    
      	\begin{frame}{Cebolla}
		Se parecen en sabor y en textura, pero esconden entre sus capas diferencias más allá de su color. Ambas tienen numerosas propiedades medicinales. En el caso de la cebolla morada, es más rica en antioxidantes y al pasarla por el cuchillo hace llorar en menor medida que algunas de sus parientas.Para la conservación de ambos tipos de cebollas, basta con almacenarlas en un lugar oscuro, fresco y seco. También se puede almacenar cortada y envuelta en plástico en el frigorífico por unos días antes de que se ponga pocha.En cuanto a la blanca, este tipo de cebolla está disponible durante todo el año y el sabor es el mismo independientemente de la estación.
		Desde el punto de vista científico, hay investigaciones médicas que señalanan que los vegetales aliáceos contienen sustancias que inhiben una variedad de tumores causantes de cáncer de mama, endometrio, colon y tracto digestivo.
	    La cebolla morada es más rica en antioxidantes que la blanca, y de ahí le viene su color rojizo. Contiene un antioxidante llamado antocianinas, que tiene efectos terapéuticos, como la reducción de las enfermedades coronarias, es antiinflamatorio, mejora la agudeza visual y el sistema inmune.
		También contiene mayor cantidad de quercetina, un potente antioxidante con un efecto antiinflamatorio y antialérgico, además favorece a la circulación sanguínea.
		Por su parte, la cebolla blanca ayuda a reducir el colesterol en la sangre (LDL o colesterol malo).
		En común, blancas y moradas, son ricas en v itamina A,B y C, calcio, magnesio, fósforo, hierro y potasio, estimulan el apetito, favorecen a la digestión, son expectorantes, bactericidas, diuréticas y depurativas.
		\end{frame}
	
	
\end{document}