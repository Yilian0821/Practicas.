\documentclass[10pt,a4paper]{article}
\usepackage[utf8]{inputenc}
\usepackage[T1]{fontenc}
\usepackage{amsmath}
\usepackage{amssymb}
\usepackage{graphicx}
\begin{document}
  \title{ Informe del Proyecto}
  \author{Yilian Vazquez Martinez}
  \date{}
  \maketitle
 
  \newpage
   \textbf{ Analisis de los Productos:}
 
  Con este proyecto se pretende analizar diferentes productos como la cebolla, la cerveza y el refresco. El objetivo principal es determinar las características y tendencias de estos productos en el mercado, con el fin de determinar patrones. Para lograr este objetivo, se van a utilizar diversas técnicas de análisis de datos y visualización, como estadísticas descriptivas, gráficos de barra, de puntos,entre otros. 
  El proyecto se llevará a cabo en Jupyter Notebook, una herramienta para el análisis de datos que permite la integración de código, visualización y texto explicativo en un solo documento. El analisis de los datos se realizara ultilizando el lenguaje de programación Python y diversas bibliotecas especializadas en ciencia de datos, como Pandas, Numpy y Matplotlib.
  Para capturar los datos de este proyecto se escogió el municipio de Guanabacoa perteneciente a la Habana. De este se visitaron 30 agros (incluído carretillas) y 30 cafeterías.El análisis de los datos recopilados, muestran una gran cantidad de información , la cual se sintetiza para lograr mostrar mediante visualizaciones, respaldadas por datos irrefutables, la respuesta a diferentes interrogantes. En el análisis a la cerveza se puede ver cuales son las marcas que están más distribuidas en Guanabacoa, de ahí se puede sacar su país de origen, los enlaces q más se utilizan, la variedad de precios y de mililitros. Al igual con la cebolla, mediante su análisis se consiguió mostrar la diferencia de oferta entre los tipos de cebollas, su información estadística en cuanto al precio, parte a la conciencia del valor de este producto en el mercado. Por último, el análisis de los refrescos gaseados, evaluando el envase y la disponibilidad de su sabor, la cual es la cola.{\normalsize }
  
  
  
   
    
\end{document}